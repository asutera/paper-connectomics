\documentclass[wcp]{jmlr}

\usepackage{enumerate}
\usepackage{bbold}
\usepackage{booktabs}
\usepackage{natbib}
\usepackage[utf8]{inputenc}
\usepackage[super]{nth}
\setlength{\textfloatsep}{15pt}

\jmlryear{2014}
\jmlrworkshop{Neural Connectomics Workshop}

\title{Simple connectome inference from partial correlation statistics in calcium imaging}

  \author{\Name{Antonio Sutera},
   \Name{Arnaud Joly},
   \Name{Vincent François-Lavet}, \Email{a.sutera@ulg.ac.be}\\
   \Name{Zixiao Aaron Qiu},
   \Name{Gilles Louppe},
   \Name{Damien Ernst}\and\Name{Pierre Geurts}
    \\
   \addr Department of EE and CS \& GIGA-R, University of Li\`ege, Belgium}

\begin{document}

\section{A large variety of performing methods}

\emph{
In the Results section "A large variety of performing methods", we intend to
assemble together various layman presentations of different approaches. If you
could provide a short popular-science style description of your approach (and
specifically of what you consider their main characterizing ingredients,
INVERSE COVARIANCE for you) this would be very helpful for me.  Here is not
really an issue of being precise, but of commenting substantial elements to a
public of... ignorant biologists! What's the logic of making a certain step?
How does the method work in a nutshell, without using equations but expressed
in plain words, etc. ? Then, based on all the texts I will collect from you, I
would build the section and let you edit it to correct mistakes or add
precisions.
}

\section{Method}
\empht{A second mini-text I ask you will converge to the Methods section. Here I need
a more detailed presentation of your method. However, it has to be precise, but
still understandable for a biologist. A bit like a recipe in a cookbook for
beginners.}

\cite{sutera2014simple}

\bibliography{references}

\end{document}
